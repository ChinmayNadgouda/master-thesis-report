In this chapter we will have a look at the 
state of the art research in the field of scene
 graph and part-object segmentation. We will have a look 
 at various papers and divide this chapter into four sections.
 The four sections are 3D scene graph generation, part-object segmentation,
 task-driven affordance grounding and open vocabulary object detection.
 Finally, we will note down the limitations in the current
 research and thus motivate our thesis to improve these limitations. 
 Specifically, we try to improve fine-grained segmentation i.e
 pat-object segmentation, also called as functionality segmentation.
 \subsection{3D Scene graph generation:}
In the subject of 3D scene graph generation we will take a look at the following papers, 
Open3DSG\cite{koch2024open3dsgopenvocabulary3dscene}, 3DSceneGraphs \cite{armeni20193d} and ConceptGraph \cite{gu2023conceptgraphsopenvocabulary3dscene}. 
We look at these papers, as to take inspiration for building our own scene graph 
using either of these implementaitons. Our take on the task two of the scenefun3D, task-driven affordance grounding will 
take help of the scene graph generated from one of these implementaitons.
To generate the scene graphs these papers utilise,
sequence of RGB-D images and poses of the camera. RGB-D images are the output from a camera.
RGB expands to Red, Green and Black, it is basically a color image and D stands for depth image. The color images
are used for object detection and the depth images along with the poses help in building the 3D point clouds. The poses are
the position and orientation of the camera in 3D. These poses can be obtained from the camera itself or by using Structure from motin softwares like
COLMAP \cite{schoenberger2016mvs} \cite{schoenberger2016sfm}. Out of the three, Concept graphs and 
Open3DSG generate the 3D scene graph using open vocabulary object detection. Open vocabulary
object detection enables the model to detect unseen objects at runtime. Concept graph uses YOLO v1.8 for this task whereas Open3DSG uses 
various vision language models (VLMs) like CLIP and Open Seg.
 The scene graphs also include the relationships between the 
objects. The relationships can be either spatial or comparative. \\
Although, these implementaitons are robust and state of the art, they face the limitation of not detecting
parts of the individual object. They lack the fine grained semantic relationships between the object and its part, for example, a door and its handle.
Such fine grained information can be useful for manipulating various obejcts in the scene. This  thesis will tackle the exact problem of 
finding fine-grained realtionships between the object and its part.

 \subsection{Part-Object segmentation:}
The task to segment out a part from a 3D object is still novel in today's research due to the limited availability of large scale 3D datasets
, works like PartNet \cite{Mo_2019_CVPR}, scenefun3D \cite{delitzas2024scenefun3d} have made notable contributions by providing benchmarks and datasets for this problem. 
However, they lack in some areas, PartNet has a limited set of data and it is not specifically related to robotics or object manipulation, the parts are not
eactly the functional element of the parent object that can be manipulated by a robot. They have general
classes of obejcts such as airplane, chair just to name a few. Whereas on the other hand, scenefun3D has parts annotated as the functional elements in the entire scene. Considering
the entire scene as the query for segmenting a part as a functional element has some limitation and their results show average 7 \% average precision, 18 \% average precison for mIoU 
50\% and 26 \% average precision for mIoU 25\%. We consider their results as a baseline and propose to improve the results. 
We plan to use only the object as the query to which the part is attached and not the entire scene, thus hypothetically improving the results.
Part Segmentation is an important task in the context of our thesis. Thus, we look at the current research to utilise one or more of these
in our final implementaiton. Works like \cite{Liu_2023_CVPR}, \cite{10.1007/978-3-031-72652-1_25} [] perform part segmentation on 2D images using techniques such as LLMs, VLMs and CNNs and then transfer the 
segments from the 2D to 3D. The reason for not directly segmenting the 3D data is that they lack large scale 3D dataset with such pre annotated part
and object point clouds. \\
However, in our work we focus on the part-object segmentaion of 3D point clouds. PointNet 2 implementation provides a way to segment out parts from a object. 
They use the PartNet dataset which has objects and its part segmented to train their 3D CNN. Similarly, we plan on using the scenefun3D dataset as well as 
the labelmaker arkit scene dataset. The labelmaker dataset provides object detections for each scene from the ARKIT dataset, where as the scenefun3D has
the parts annotated for around 700 scenes for the same ARKIT scenes. We plan to merge the two datasets to get the final data which contains the objects as 
well as their parts annotated. We further plan to use this dataset to train Point Net ++ and Mask3D.
 \subsection{Task-driven affordance grounding:}
Given a specific query (task), identifying and associating the functional elements of an object to the task in the query is known as Task-driven affordance grounding. 
In the context of grounding 3D scene affordances to the given task, \citet{liu2024grounding3dsceneaffordance} propose
 Egocentric Interaction-driven 3D Scene Affordance Grounding (Ego-SAG). The paper proposes to identify affordance regions in a 3D scene on the basis of 
 an egocentric vieo of the interaction. The paper further compares their results with the following line of work,
\cite{huang2024openins3dsnaplookup3d}, \cite{Nguyen_2024_CVPR} and \cite{takmaz2023openmask3dopenvocabulary3dinstance}. However, the fine-grained 
segmentation of the functional element required to perform the task is absent. Open3dIS presents a framework for 3D instance segmentation which is based on
zero shot leraning for unseen objects. They leverage 2D mask for guding segmentation process and integrate it with 3D feature extraction pipelines. And, in the context
of task-driven affordance grouding, the capabilities of Open3DIS are highly relevant but lack the depth of fine grained segmentation. Open Mask 3D is a framework
 for open-vocabulary 3D scene understanding. They employ
various foundation models to generalize various 3D tasks. Transformer based backbone is leveraged that integrates 2D and 3D features to enhance 
recognition and segmentation tasks. Therefore, tasks such as identifying objects capable of specific interactions benefit from OpenMask 3D. 
But even OpenMask3D lacks finegrained scene understanding at the functional element level
Open Mask3D \cite{takmaz2023openmask3dopenvocabulary3dinstance} is also used for comparison by
\citet{delitzas2024scenefun3d}, here they modify the implementaiton of OpenMask3D a bit to incorporate segmentaion of the functional elements. 
They describe this task in \cite{delitzas2024scenefun3d} and present their results and show notable improvements.\\
However, the results obtained by \citet{delitzas2024scenefun3d}
can be considered as baseline results and further improved by utilisig scene graph as an intermediate step for task-driven affordance grounding.

 \subsection{Open-vocabulary object detection:}
Open vocabulary object detection refers to the ability of a model to detect obejcts not seen during the training process. 
In short, the model can detect an label unseen objects at runtime.This is useful for real-world applications where annotations for all the 
existing objects in the world are not feasible or when there are constraints on resources whilst training. State of the art open vocabulary object detections include
YOLO world, an advanced, realtime Ultralytics YOLOv8 based approach. It enables detection fo any object in a given image on the basis of descriptive texts \cite{cheng2024yolow}. \\

In this thesis, we mitigate all the abovementioned limitations and propose
a novel way to create scene graphs that have the ability to functionally segment
interactive elements and provide task-driven affordance grounding keeping in context the 
segmented fine-grained segmentations. At the core of our idea is the 
point cloud network which will segment the fine-grained part from its object. For this we will train pre existing 
models like PointNet ++ and Mask3D on our own merged dataset obtained from merging scenefun3D and labelmaker
ARKIT scene datasets. Further, we will employ a LLM to assign captions with semantically rich descriptions and 
labels for each instance of the part segment 
obtained out of every object and store them for language querying. 
To summarise, our novel contributions are:
\begin{compactenum}[.]
\item We propose a novel way to build scene graphs which also includes fine grained segmentation of objects and their parts by tackling the below two tasks.
\item We tackle the Task 1 functionality segmentation from the authors \citet{delitzas2024scenefun3d} and propose to improve their 
baseline results.
\item We also tackle the Task2 task-driven affordance grounding from the authors mentioned in the above point 
and propose to improve on the concept by leveraging scene graphs.
\end{compactenum}


Maintain coherency, if you use functional elements use that. if u use part-obejct use that if you use fine grained segs use that. Dont mix all!
