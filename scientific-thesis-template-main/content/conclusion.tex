\subsection{Future Work:}
The system proposed is not full-proof and has some limitations. The limitations come from the fact that the size of the dataset used to train the models (Mask3D 
and PointNet++) was small. Another point that can be improved in out work is the use of newer models in the future that are designed to segment out fine grained masks
for such small interactive elements. Apart from these limitations there are some future directions that our work can take forward.
\begin{compactenum}[1.]
    \item	Query the system using voice commands, a voice to text recognition model can be implemented to tranform the commands given by voice by a human to text. This text can then be fed to our system
    \item	The part-object detection models can be used as separate components and be used in other downstream tasks which require segmenting part from an object such as bin picking.
    \end{compactenum}
In this thesis, we present a novel way to extend the implmentation of concept graphs that enables the generation of scene graphs with adiitional fine grained 
semantically rich information of functionally interactive elements like 'door handles', 'door knobs', and 'sink faucets'. To achieve this goal, models like Mask3D,
Point net were leverage and trained on datasets like scenefun3d and label maker enabled the successful implmentation of this Research topic. Further, we also evaluated 
our system based on the baseline results of the two tasks defined by scenefun3d and saw considerable improvement of over 5\% in AP. We envision further improvements in
our model with newer 3D datasets for segmentation tasks and more capable models. 

