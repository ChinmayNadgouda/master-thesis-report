In the chapter, we will look at the methodology used to realise the milestones set in \cref{chap:k1}. We first list the approach
followed during the realisation of this thesis. Next, we propose a solution for the limitations we saw in \cref{chap:k2} and then
we give a detailed description of the system design and its main components. Finally, we deep dive into the implementation details.

% \section{Research Approach:}
% \begin{compactenum}[1.]
%     \item Implement the state-of-the-art scene graph generation framework, ConceptGraphs.
%     \item Gather dataset by recording scenes manually in the SIR lab.
%     \item Verify the implementation using pre-recorded, synthetic and captured datasets.
%     \item Extend the implementation by including a novel way to segment part from an object in the existing ConceptGraphs
%     \item Test multiple methods to segment part from an object
%     \item Evalute methods and feasibility
%     \item Integrate the method into current implementation
%     \item Test and Validate the new implementation after integration
%     \item Present Results using Average Precision metrics and Snapshots of the output


\section{Approach}
The study utilizes a systematic approach to create, enhance, and evaluate a framework for scene graph generation. The subsequent are the essential steps:
\begin{compactenum}[1.]
\item \textbf{Implement the ConceptGraphs:} Utilize ConceptGraphs, a state-of-the-art framework for scene graph generation.
\item \textbf{Dataset Collection:} Capture a real-world dataset by recording real-world scenes within the SIR laboratory.
\item \textbf{Execution:} To ensure accuracy and correctness, verify the implementation with pre-recorded and synthetic datasets.
\item \textbf{Part-Object segmentation:} Develop and test an innovative method for segmenting part and object within the ConceptGraphs framework.
\item \textbf{Incorporating ConceptGraphs:} Integrate the segmentation method into the existing ConceptGraphs implementation.
\item \textbf{Verification of Part-Object Segmentation:} To guarantee the enhanced implementation's accuracy and reliability, do comprehensive testing.
\item \textbf{Evaluation and Examination: } Assess several segmentation methodologies based on their average precision and feasibility.
\item \textbf{Results: } Use Average Precision (AP) metrics to evaluate the completed system, and illustrate  qualitative results with the help of output snapshots.
\end{compactenum}